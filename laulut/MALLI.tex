\addcontentsline{toc}{section}{\arabic{songnum} MALLILAULU}
\beginsong{MALLILAULU}[by={san. SANOITTAJA-TÄHÄN säv. SÄVELTÄJÄ-TÄHÄN},
index={ALKUSANOJA-TÄHÄN}
]


% TAHTILAJI TÄHÄN:
\meter{4}{4}


%********** INTRO **********
\ifchorded
\beginverse*
{\nolyrics
|\[Em] \[Em/D] |\[Cmaj7] \[H7] \rep{2}
}
\endverse
\fi


%********** 1. SÄKEISTÖ **********
% Säkeistön soinnut tallennetaan nimellä "verse_oma".
\beginverse\memorize[verse_oma]
Hän |\[Em]vaeltaa läpi |\[Am]kaupungin,
hän |\[Am/F#]silmin pimein \[H7]etsii, ei \brk |\[Em]löydä vain.
\endverse


%********** 1. SÄKEISTÖN KERTOSÄE **********
\beginchorus\memorize[chorus_oma]
Ja |\[Am/F#]polte, joka \[H7]sieluansa |\[Em]korven\[Em/D]taa,
se on |\[Cmaj7]vain, \[H7]vain rakka|\[Em]us. \[Em/D] 
\endchorus


%********** 2. SÄKEISTÖ **********
% Säkeistön soinnut on tallennettuna nimellä "verse_oma".
% Ne otetaan käyttöön käskyllä \replay{}, ja ^-merkeillä.
\beginverse\replay[verse_oma]
Hän |^kuuntelee vain |^varjoja,
ja |^nälkänsä hän ^unelmillaan \brk |^sammuttaa.
\endverse

%********** 2. SÄKEISTÖN KERTOSÄE **********
\beginchorus\replay[chorus_oma]
Ja |^polte, joka ^sieluansa |^korven^taa,
se on |^vain, ^vain rakka|^us. ^
\endchorus

%********** KITARAOTTEET **********
% Joitain otteita on tallennettuna kansioon "soinnut".
% Lisää voi tehdä itse käskyllä \gtab{}{}.
\ifchorded
\vspace{\versesep} % Tämä rivi luo välin kappaleen ja sointuotteiden väliin.
\noindent % Ei sisennystä.
\gtab{Em}{1:022000:023000}
\gtab{Am}{1:X02210:002310}
\gtab{Am/F#}{1:202210:203410}
\gtab{H7}{1:X21202:021304}
\fi


\endsong


%********** SEKALAISIA OHJEITA **********
%
%********** % **********
% Kaikki, mitä kirjoitetaan prosenttimerkin jälkeen
% samalle riville, jää kokonaan huomioitta lopulli-
% sessa dokumentissa.
%
%********** \brk **********
% \brk-käskyllä merkitään haluttu rivinvaihdon kohta.
% Jos rivi on liian pitkä lopullisessa dokumentissa,
% se katkeaa \brk:n kohdalta.
%
%********** \memorize[NIMI] **********
% Tallentaa soinnut tästä eteenpäin nimellä NIMI.
%
%********** \replay[NIMI] **********
% Ottaa NIMI-nimellä tallennetut soinnut käyttöön.
% Merkki ^ valisee seuraavan soinnun tallennetusta
% sointujonosta.