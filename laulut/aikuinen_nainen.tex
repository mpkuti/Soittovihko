\addcontentsline{toc}{section}{\arabic{songnum} Aikuinen nainen (kesken)}

\beginsong{Aikuinen nainen (kesken)}[by={suom. san. K. Liuhala säv. G. Savio--A. Gasalle},
index={Aikuinen nainen mä oon}
]

\meter{6}{8}


%-------------------------------------------------------------------------------
% INTRO
%-------------------------------------------------------------------------------
\beginverse*
|\[F] \[Fsus] |\[F] \[Fsus]
|\[F]Aikuinen \[Fsus]nainen mä oo|\[F]n, \[Fsus] 
|\[F]en enää eksy \[Dm]maailman |\[Gm]tuulii\[Gsus]n.
|\[Gm]Liittomme \[Gsus]vahvistukoo|\[Gm]n,
\[Bb]tahtoisin |\[F]jatkaa \[F]hyvin alkanutta |\[C]matkaa.
\endverse
%-------------------------------------------------------------------------------


%-------------------------------------------------------------------------------
% 1. SÄKEISTÖ
%-------------------------------------------------------------------------------
\beginverse
\[D7]Kun aika \memorize[verse_oma]|\[G]vaikeudet \[Gsus]tiellemme tu|\[G]o, \[Gsus]
|\[G]onnesta me teemme \[Em]suojaavan |\[Am]muuri\[Asus]n.
|\[Am]Voimaa tää \[Asus]rakkaus |\[Am]suo. \[C]Onnemme
|\[G]kestää, \[G]minkään emme anna |\[D]estää.
\[D]Paljosta lu|\[C]ovun, \[D7]jos saan luonain pitää
|\[G]sun. \[Em] |\[Am7] \[D7]
\endverse

\beginchorus\memorize[chorus_oma]
|\[G]E\[Em]|\[Am7]n \[D7]vapauttas tahdo
|\[G]riis\[Hm7]tä|\[Am7]ä, \[D7] tahdo
|\[G]e\[H7]n|\[Em] \[A7]itsenäisyyttäsi |\[Dsus]ki\[Dsus]is|\[D]tää.
\[D]Sua \[C]tar\[D7]vit|\[G]s\[G]e|\[C]n \[D7]sydänystäväksi
|\[G]ai\[H7]kuisen |\[Em]naisen.
\[C]Kaiken me |\[G]jaamme,
\[G]toisiamme tuke|\[D]kaamme.
\[D]Aikuinen |\[C]nainen \[D7]tuntee arvon rakkau|\[G]den. \[Gsus] |\[G] \[Gsus]
\endchorus
%-------------------------------------------------------------------------------

\beginverse\replay[verse_oma]
|^Selkääni käännä mä en
onnestani taistelen enkä pelkää
toisista huolehtien
kestämme paineet
meitä eivät kaada laineet
tuulilta suojaan sinut vien
jos viedä saan
\endverse

En vapauttas tahdo riistää
tahdo en itsenäisyyttäsi kiistää
sua tarvitsen sydänystäväksi
aikuisen naisen
kaiken me jaamme
toisiamme tukekaamme
aikuinen nainen tuntee arvon rakkauden

Sua tarvitsen sydänystäväksi
aikuisen naisen
kaiken me jaamme
toisiamme tukekaamme
aikuinen nainen tuntee arvon rakkauden.. (x2)


%********** 2. SÄKEISTÖ **********
% Säkeistön soinnut on tallennettuna nimellä "verse_oma".
% Ne otetaan käyttöön käskyllä \replay{}, ja ^-merkeillä.
%\beginverse\replay[verse_oma]
%Hän |^kuuntelee vain |^varjoja,
%ja |^nälkänsä hän ^unelmillaan \brk |^sammuttaa.
%\endverse

%********** 2. SÄKEISTÖN KERTOSÄE **********
%\beginchorus\replay[chorus_oma]
%Ja |^polte, joka ^sieluansa |^korven^taa,
%se on |^vain, ^vain rakka|^us. ^
%\endchorus

%********** KITARAOTTEET **********
% Joitain otteita on tallennettuna kansioon "soinnut".
% Lisää voi tehdä itse käskyllä \gtab{}{}.
\ifchorded
\vspace{\versesep} % Tämä rivi luo välin kappaleen ja sointuotteiden väliin.
\noindent % Ei sisennystä.
\gtab{F}{1:133211:134211}
\gtab{Fsus4}{1:133311:123411}
\gtab{Dm}{1:XX0231:000231}
\gtab{Gm}{1:355333:134111}
\gtab{Gsus4}{1:355533:123411}
\gtab{Bb}{1:X13331:012341}
\gtab{C}{1:X32010:032010}
\gtab{D7}{1:XX0212:0000213}
\gtab{Am}{1:X02210:002310}
\gtab{Asus4}{1:X02230:002340}
\gtab{Em}{1:022000:023000}
\gtab{G}{1:320003:210003}
\gtab{Gsus4}{1:3X0013:300014}
\gtab{Am7}{1:X02010:002010}
\fi


\endsong


%********** SEKALAISIA OHJEITA **********
%
%********** % **********
% Kaikki, mitä kirjoitetaan prosenttimerkin jälkeen
% samalle riville, jää kokonaan huomioitta lopulli-
% sessa dokumentissa.
%
%********** \brk **********
% \brk-käskyllä merkitään haluttu rivinvaihdon kohta.
% Jos rivi on liian pitkä lopullisessa dokumentissa,
% se katkeaa \brk:n kohdalta.
%
%********** \memorize[NIMI] **********
% Tallentaa soinnut tästä eteenpäin nimellä NIMI.
%
%********** \replay[NIMI] **********
% Ottaa NIMI-nimellä tallennetut soinnut käyttöön.
% Merkki ^ valisee seuraavan soinnun tallennetusta
% sointujonosta.