\addcontentsline{toc}{section}{\arabic{songnum} Kulkuri ja joutsen}
\beginsong{Kulkuri ja joutsen}[by={san. Reino Helismaa säv. Lasse Dahlquist},
index={Oon kulkuriksi syntynyt mä vainen}
]

%% Säkeistöjen sisennys:
\setlength{\versenumwidth}{0mm}

% TAHTILAJI
\meter{4}{4}

%********** 1. SÄKEISTÖ **********
% Säkeistön soinnut tallennetaan nimellä "verse_oma".
\beginverse\memorize[verse_oma]
Oon |\[C]kulkuriksi syntynyt mä |\[C]vainen
ja |\[C]paljain jaloin \[Em7]kierrän \[A7]maail|\[Dm7]maa. \[G7]
Näin |\[Dm7]kerran unta: joutsen taiva|\[G7]hainen
mun |\[Dm7]antoi hetken \[G7]kanssaan taival|\[C]taa.
Näin |\[C]maat ja metsät, järvet maani |\[C]armaan,
sen |\[Gm7]kaiken sini\[C7]taivahalta |\[F]näin. \[A7]
Tuuli |\[Dm7]lauloi laulu\[D#\(^\circ\)]jaan,
sadun |\[C]hohde \[B&7]peitti \[A7]maan,
\[A7]iha|\[Dm7]nampaa en mä \[G7]koskaan nähdä |\[G7-9]saa\[C]ta.
\endverse

%********** 2. SÄKEISTÖ **********
\beginverse\replay[verse_oma]
Ja |^laulun tämän lauloi mulle |^tuuli:
"On |^kaikkein kaunein ^aina ^oma |^maa. ^
Niin |^moni muuta paremmaksi |^luuli,
mut |^pettymyksen ^itsellensä |^saa.
Ei |^missään taivas sinisemmin |^loista,
ei |^missään hanki ^hohda kirkkaam|^min. ^
Tämä |^paina sydä^mees,
niin on |^taivas ^aina ^sees
^ja sun |^taipaleeltas ^murhe kauas |^kaik^koo."
\endverse   

%********** 3. SÄKEISTÖ **********
\beginverse\replay[verse_oma]
Niin |^päättyi uni, mutta vielä |^vuotan
mä |^kohtaavani ^kerran ^joutse|^nen, ^
ja |^siihen asti unelmiini |^luotan;
on |^helppo unek^sia ihmi|^sen.
Siks |^paljain jaloin onneani |^etsin
ja |^huolet laulu^llani haihdu|^tan. ^    
Elä|^mä on iha^naa,
kun sen |^oikein ^oival^taa
^ja kun |^lentää siivin ^valkein niin kuin |^jout^sen.
\endverse

\beginverse*
Elä|\[F]mä on iha\[\Ferli{\bfseries\ttfamily F}#\(^\circ\)]naa,        
kun sen |\[C]oikein \[B&7]oival\[\Ferli{\bfseries\ttfamily A7}]taa
ja kun |\[Dm7]lentää siivin \[G7]valkein niin kuin |\[\Ferli{\bfseries\ttfamily G7-9}]jout\[\Ferli{\bfseries\ttfamily C}]sen.
\endverse

%********** KITARAOTTEET **********
% Joitain otteita on tallennettuna kansioon "soinnut".
% Lisää voi tehdä itse käskyllä \gtab{}{}.
\ifchorded
%\vspace{\versesep} % Tämä rivi luo välin kappaleen ja sointuotteiden väliin.
%\noindent % Ei sisennystä.
%\gtab{Em}{1:022000:023000}
%\gtab{Am}{1:X02210:002310}
%\gtab{Am/F#}{1:202210:203410}
%\gtab{H7}{1:X21202:021304}
\fi


\endsong


%********** SEKALAISIA OHJEITA **********
%
%********** % **********
% Kaikki, mitä kirjoitetaan prosenttimerkin jälkeen
% samalle riville, jää kokonaan huomioitta lopulli-
% sessa dokumentissa.
%
%********** \brk **********
% \brk-käskyllä merkitään haluttu rivinvaihdon kohta.
% Jos rivi on liian pitkä lopullisessa dokumentissa,
% se katkeaa \brk:n kohdalta.
%
%********** \memorize[NIMI] **********
% Tallentaa soinnut tästä eteenpäin nimellä NIMI.
%
%********** \replay[NIMI] **********
% Ottaa NIMI-nimellä tallennetut soinnut käyttöön.
% Merkki ^ valisee seuraavan soinnun tallennetusta
% sointujonosta.