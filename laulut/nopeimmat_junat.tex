\addcontentsline{toc}{section}{\arabic{songnum} Nopeimmat junat}
\beginsong{Nopeimmat junat}[by={Anna Puu},
index={Olen kasvanut kiinni maahan}
]

\capo{3}

% TAHTILAJI TÄHÄN:
\meter{4}{4}


%********** INTRO **********
\ifchorded
\beginverse*
{\nolyrics
|\[F] |\[G] |\[F] |\[G]
|\[F] |\[G] |\[C] |\[C]
}
\endverse
\fi


%********** 1. SÄKEISTÖ **********
% Säkeistön soinnut tallennetaan nimellä "verse_oma".
\beginverse\memorize[verse_oma]
|\[F] Olen kasvanut |\[G]kiinni maahan, \brk|\[F] lapsuuden |\[G]maisemaan.
|\[F] Olen kasvanut |\[G]kiinni poikaan,  \brk|\[C]siihen yhteen ja |\[Dm]oikeaan.
|\[F] Olen kasvanut |\[G]kiinni taloon, \brk|\[F] jonka ikkunan|\[G]pielistä vetää.
|\[F] En tahdo |\[G]kuuta taivaal|\[C]ta \brk kuin kerran |\[Dm]vuoteen |\[F]enää.
Paikka se on |\[G]tämäkin.
\endverse


%********** 1. SÄKEISTÖN KERTOSÄE **********
\beginchorus\memorize[chorus_oma]
|\[C] Pilvien |\[C]alla, maan |\[Dm]päällä, |\[Dm]
|\[Em] mutta nopeimmat |\[Em]junat eivät, \brk|\[F] pysähdy |\[G]enää täällä.
|\[C] Pilvien |\[C]alla, maan |\[Dm]päällä, |\[Dm]
mutta |\[F]nopeimmat \[Em]junat, |\[Dm]nopeimmat \[C]junat,
ei |\[Am]nopeimmat \[G]junat enää  \mbar{2}{4}\[F\tiny(1/2)]pysähdy  \mbar{4}{4}\[C]täällä. |\[C]
\endchorus

%********** INTERLUDE **********
\ifchorded
\beginverse*
{\nolyrics
|\[F] |\[G] |\[F] |\[G]
|\[F] |\[G] |\[C] |\[C]
}
\endverse
\fi

%********** 2. SÄKEISTÖ **********
% Säkeistön soinnut on tallennettuna nimellä "verse_oma".
% Ne otetaan käyttöön käskyllä \replay{}, ja ^-merkeillä.
\beginverse\replay[verse_oma]
|^ Olen kasvanut |^kiinni lähtöön, \brk|^ kaupungin |^kaipuuseen.
|^ Lähtenyt |^monta kertaa, \brk|^palannut aina |^entiseen.
|^ Olen kasvanut |^kiinni talveen, \brk|^ sen kuristus|^otteeseen
|^ Kun pysyn sun |^luonas tiedän, \brk|^vielä jaksan |^kevääseen
|^ Paikka se on |^tämäkin
\endverse

%********** 2. SÄKEISTÖN KERTOSÄE **********
\beginchorus\replay[chorus_oma]
|\[C] Pilvien |\[C]alla, maan |\[Dm]päällä, |\[Dm]
|\[Em] mutta nopeimmat |\[Em]junat eivät, \brk|\[F] pysähdy |\[G]enää täällä.
|\[C] Pilvien |\[C]alla, maan |\[Dm]päällä, |\[Dm]
mutta |\[F]nopeimmat \[Em]junat, |\[Dm]nopeimmat \[C]junat,
ei |\[Am]nopeimmat \[G]junat enää  \mbar{2}{4}\[F\tiny(1/2)]pysähdy  \mbar{4}{4}\[C]täällä. |\[C]
\endchorus

%********** INTERLUDE **********
\ifchorded
\beginverse*
{\nolyrics
|\[F] \[Em] |\[Dm] \[C] |\[Am] \[G]  \mbar{2}{4}\[F\tiny(2/4)]  \mbar{4}{4}\[C] |\[C]
}
\endverse
\fi

%********** BRIDGE **********
\beginverse*\replay[verse_oma]
|^ Olen kasvanut |^kiinni maahan, \brk|^lapsuuden |^maisemaan. |
\endverse

%********** 3. KERTOSÄE **********
\beginchorus\memorize[chorus_oma]
|\[C] Pilvien |\[C]alla, maan |\[Dm]päällä, |\[Dm]
|\[Em] mutta nopeimmat |\[Em]junat eivät, \brk|\[F] pysähdy |\[G]enää täällä.
|\[C] Pilvien |\[C]alla, maan |\[Dm]päällä, |\[Dm]
mutta |\[F]nopeimmat \[Em]junat, |\[Dm]nopeimmat \[C]junat,
ei |\[Am]nopeimmat \[G]junat enää  \mbar{2}{4}\[F\tiny(1/2)]pysähdy  \mbar{4}{4}\[C]täällä, |\[C]
\endchorus

%********** OUTRO **********
\beginverse*
mutta |\[F]nopeimmat \[Em]junat, ei |\[Dm]nopeimmat \[C]junat,
ei |\[Am]nopeimmat \[G]junat enää |\[F\tiny(1/2)]pysähdy |\[C]täällä. |\[C]
\endverse

%********** KITARAOTTEET **********
% Joitain otteita on tallennettuna kansioon "soinnut".
% Lisää voi tehdä itse käskyllä \gtab{}{}.
%\ifchorded
%\vspace{\versesep} % Tämä rivi luo välin kappaleen ja sointuotteiden väliin.
%\noindent % Ei sisennystä.
%\gtab{Em}{1:022000:023000}
%\gtab{Am}{1:X02210:002310}
%\gtab{Am/F#}{1:202210:203410}
%\gtab{H7}{1:X21202:021304}
%\fi


\endsong


%********** SEKALAISIA OHJEITA **********
%
%********** % **********
% Kaikki, mitä kirjoitetaan prosenttimerkin jälkeen
% samalle riville, jää kokonaan huomioitta lopulli-
% sessa dokumentissa.
%
%********** \brk **********
% \brk-käskyllä merkitään haluttu rivinvaihdon kohta.
% Jos rivi on liian pitkä lopullisessa dokumentissa,
% se katkeaa \brk:n kohdalta.
%
%********** \memorize[NIMI] **********
% Tallentaa soinnut tästä eteenpäin nimellä NIMI.
%
%********** \replay[NIMI] **********
% Ottaa NIMI-nimellä tallennetut soinnut käyttöön.
% Merkki ^ valisee seuraavan soinnun tallennetusta
% sointujonosta.